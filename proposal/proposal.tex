\documentclass[a4paper]{article}

\usepackage[english]{babel}
\usepackage[utf8]{inputenc}
\usepackage{amsmath}
\usepackage{amssymb}
\usepackage{mathtools}
\usepackage{mathabx}
\usepackage{mathrsfs}
\usepackage{titlesec}
\usepackage{xcolor}
\usepackage{enumitem}
\usepackage[autostyle]{csquotes}
\usepackage[backend=biber,style=alphabetic]{biblatex}
\usepackage{hyperref}
\usepackage{cleveref}
\usepackage{subfig}
\addbibresource{proposal.bib}

% format subsections
\titleformat{\subsection}[block]{\hspace{1em}}{\bfseries\thesubsection}{1em}{\bfseries}
% support vertical lines in matrices
\makeatletter
\renewcommand*\env@matrix[1][*\c@MaxMatrixCols c]{%
  \hskip -\arraycolsep
  \let\@ifnextchar\new@ifnextchar
  \array{#1}}
\makeatother
% angled fractions
\newcommand*\rfrac[2]{{}^{#1}\!/_{#2}}
% text below matrices
\newcommand*{\putunder}[2]{%
  {\mathop{#1}_{\mathstrut #2}}%
}
% multiline comment
\newcommand{\comment}[1]{}
% vertical bars
\newcommand{\norm}[1]{\left\lVert#1\right\rVert}

\title{\vspace{-2cm}Bachelor Project on Local Differential Privacy}

\author{Johan Ringmann Fagerberg (jofag17) - 1996-09-24}
\date{}

\begin{document}

\maketitle

\begin{description}[style=nextline]
    \item[Motivation] Differential privacy is a recently proposed measure of the level of privacy provided by random functions. It provides a formal measure for how much information a potential attacker can achieve from a perturbed data set, when the data perturbation consists of random noise.
    
    In global differential privacy, a global aggregator has access to the real data prior to perturbing it. This differs from local differential privacy, where users perturb their data before sending it to the aggregator.
    
    The goal of this bachelor project is to gain an understanding of 
differential privacy in general, with a particular focus on local differential 
privacy, and attempt to use that to produce an architecture for a trust-less 
and privacy-respecting web analytics solution.

\item[Plan]
    The project will consist in 3 phases ad detailed below
    \begin{itemize}
     \item Survey of the state-of the art (tentatively January-February). 
The survey will start by consulting \cite{desfontain_overview}, i.e., a reading 
list of key papers in differential privacy presented as a semi-technical blog 
series on differential privacy. Moreover, we will also read 
\cite{localdiffpriv_survey}, i.e., a recent survey on local different privacy 
methods for various use cases.
\item
Development of a prototype (tentatively mid February-April). A web analytics 
system à la \cite{webanalytics_2012} will be consider for developing a 
prototype that uses local differential privacy
% (or potentially \href{
% https://desfontain.es/privacy/local-global-differential-privacy.html#the-best-of
% -both-worlds}{ESA}
to remove the need for users to trust a potentially malicious 
third party. The prototype will be tested to evaluate computational overhead on both client and server, as well as bandwidth overhead in terms of communicating the perturbed data from client to server.
\item
Report writing (tentatively April-May). The final report will be written 
following the academic conventions.
    \end{itemize}
    
    \item[Risks] There is a possibility that it would be impossible to generate 
an actually useful trust-less web analytics solution (e.g. due to increased 
noise from local differential privacy).

\item[Outcomes]
At the end of the project, the outcomes of the project will be
\begin{itemize}
\item A report written in English and following the standard academic writing
    conventions. The reprort will contain the description of the 
state-of-the-art, the description of the prototype and its preliminary 
evaluation.
\item The code of the prototype and the data for the experiments performed
\end{itemize}
   
\end{description}

\printbibliography

\end{document}
